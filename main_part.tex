\documentclass[a4paper,12pt,oneside]{extarticle}
%характеристики: 12-ый стандартный шрифт, размер листа А4

% Внимание! Чтобы код ниже работал, необхоидмо переключиться на движок XeLaTe
\usepackage{fontspec} % подключаем модуль для работы со шрифтами
\PassOptionsToPackage{no-math}{fontspec} %  https://tex.stackexchange.com/a/26295/104425
\usepackage{polyglossia} % Поддержка многоязычности
\usepackage{ragged2e} % патек для нормального выравнивания текста
\usepackage{svg}
\usepackage[ % задаём макет документа
left=2.8cm, % отступ от левого края 
right=1.3cm, % отступ от правого края
top=2.0cm, % отсутп сверху
bottom=1.59cm, % отступ снизу
%paperheight=11.69in, % высота листа
%paperwidth=8.5in% % ширина листа
]{geometry} 

% настройка шрифтов и языков
\defaultfontfeatures{Ligatures=TeX,Mapping=tex-text}

\setmainlanguage[babelshorthands = true]{russian} % устанавливаемя русский как основной язык
\setotherlanguage{english} % устанавливаем английский как дополнительный язык

\defaultfontfeatures[ArrialNarrow]{ % подключаем шрифт из файлов
	Path = fonts/,
	Extension = .ttf,
	UprightFont = arialnarrow,
	BoldFont = arialnarrow_bold,
	ItalicFont = arialnarrow_italic,
	BoldItalicFont = arialnarrow_bolditalic,
}

\setmainfont{Times New Roman} % устанавливаем главный шрифт 
\setsansfont[Ligatures=TeX]{ArialNarrow} % устонавливаем дополнительный шрмфт 

\newfontfamily\cyrillicfont[Script=Cyrillic]{Times New Roman} % делаем так, чтобы русский язык мог отображаться главным шрифтом 
\newfontfamily\cyrillicfontsf[Script=Cyrillic]{ArialNarrow} % делаем так, чтобы русский язык мог отображаться дополнительным шрифтом 

\newfontfamily\englishfont{Times New Roman} % делаем так, чтобы английский язык мог отображаться главным шрифтом 
\newfontfamily\englishfontsf{ArialNarrow} % делаем так, чтобы английский язык мог отображаться дополнительным шрифтом 

\pagenumbering{gobble} % убирает нумерацию
% Установить пакет ttf-xits для математики!

\usepackage{xltxtra,fontspec,amsmath,unicode-math}



\textwidth=175mm
\textheight=260mm
\oddsidemargin=-.4mm % отступ справа
\headsep=5mm% отступ сверху

\topmargin=-1in
\unitlength=1mm

\def\VL{\line(0,1){15}}% это относится к рамке поле снизу вертикальные
\def\HL{\line(1,0){185}}
\def\Box#1#2{\makebox(#1,5){#2}}
\def\simpleGrad{\sl\small\noindent\hbox to 0pt{%
		\vbox to 0pt{%
			\noindent\begin{picture}(185,287)(5,0)
				\linethickness{0.5mm}% ширина линий рамки
				\put(0,0){\framebox(185,287){}}
				\put(0,0){\Box{8}{Лит.}}
				\put(0, 15)\HL
				\multiput(0, 5)(0, 5){2}{\line(1,0){65}}
				\put(7, 0){\VL\Box{10}{Изм.}}
				\put(17, 0){\VL\Box{23}{\No~докум.}}
				\put(40, 0){\VL\Box{15}{\bfseries Подп.}}
				\put(55, 0){\VL\Box{10}{Дата}}
				\put(65, 0){\VL\makebox(110,15){\large\sc\rightmark}}
				\put(175, 0){\VL\makebox(10,10){\normalsize\thepage}}
				\put(175,10){\line(1,0){10}}
			\end{picture}
}}}

\makeatletter
\def\@oddhead{\simpleGrad}
\def\@oddfoot{}
\makeatother
\begin{document}
	\section{ВВЕДЕНИЕ}
	\begin{flushleft}
		В последнее десятилетие нейронные сети пользуются большим спросом и привлекают немалое внимание публики, которая даже не понимает, что это такое. И хотя складывается впечатление, что нейронные сети — это новая и крайне прогрессивная технология, в действительности это одно из старейших направлений в информатике/кибернетике, сопоставимое по возрасту с операционными системами и гораздо старше, чем такие обыденные вещи, как работа с графикой и даже обычный графический интерфейс пользователя.
	\end{flushleft}
	
	\begin{flushleft}
		И несмотря на то, что нейронные сети появились довольно давно, в действительности их потенциал стал раскрываться лишь в последнее десятилетие, когда стали развиваться такие вещи (приборы? Просто странная формулировка получается), как видеокарты и кратно возросла мощность компьютеров. В последние 2 года нейронные сети совершили как качественный, так и количественный скачок. Результатом чего явилась возможность использовать нейронные сети в таких, традиционно человеческих сферах, как рисование, генерирование музыки, ответы на вопросы человека с поразительной точностью ответов и многое другое.
	\end{flushleft}
	
	\begin{flushleft}
		И хотя большая часть восторгов широкой публики направлена на вышеописанные способы применения, существует не мало сфер применения, которые ускользают от взора широкой публики. Одной из таких сфер применения является промышленность. 
	\end{flushleft}
	
	\begin{flushleft}
		Применение нейронных сетей в промышленности является потенциально полезной задумкой. Тем не менее, современные нейронные сети пишутся на языках высокого уровня и непригодны для использования в промышленности. Для полноценного внедрения нейронных сетей в промышленность, необходимо писать их на языках более низкого уровня, чем самый популярный язык для нейронных сетей python.
	\end{flushleft}
	
	\newpage
	
	\section{АНАЛИЗ ПРЕДМЕТНОЙ ОБЛАСТИ И ПОСТАНОВКА ЗАДАЧ}
	
	\subsection{Основные понятия предметной области}
	
	\begin{flushleft}
		\bfseries Проблематика
	\end{flushleft}
	\begin{flushleft}
		Мероприятия по мойке оборудования, контактирующего с продуктами, являются важной частью производства пищевых продуктов. Необходимо помнить, что производители пищевых продуктов всегда обязаны выдерживать высокие гигиенические стандарты.
	\end{flushleft}
	
	\begin{flushleft}
		Это применимо как к оборудованию, так и, конечно же, к персоналу, участвующему	в производстве. Эти обязательства можно разбить на следующие три группы:
	\end{flushleft}
	
	\begin{enumerate}
		\item Коммерческие обязательства
		\item Моральные обязательства
		\item Юридические обязательства.
	\end{enumerate} 
	
	\begin{flushleft}
		\bfseries Коммерческие обязательства
	\end{flushleft}
	
	\begin{flushleft}
		Качественный, полезный продукт с высокой сохраняемостью, отвечающий санитарно-
		гигиеническим требованиям, безопасный для здоровья, безусловно, привлекателен для
		потребителя и будет неоднократно им приобретаться. Если продукт содержит загрязнения,
		плохо сохраняется или является объектом жалоб, есть возвраты продукта, он имеет
		соответственно плохую репутацию.
	\end{flushleft}
	
	\begin{flushleft}
		Следует всегда принимать во внимание потенциальные последствия
		неудовлетворительной мойки, низких стандартов и низкого качества.
	\end{flushleft}
	
	\begin{flushleft}
		\bfseries Моральные обязательства
	\end{flushleft}
	
	\begin{flushleft}
		В основном потребители продуктов никогда не видели завода-производителя и условий
		обработки продукта. Они доверяют фирме, ее репутации и полагают, что все операции
		проводятся в условиях высокой санитарии, хорошо подготовленным персоналом, который
		постоянно отслеживает эти факторы.
	\end{flushleft}
	
	\begin{flushleft}
		\bfseries Юридические обязательства
	\end{flushleft}
	\begin{flushleft}
		Закон старается защитить заказчика и покупателя в отношении здоровья и качества.
		Невыполнение законодательных норм, местных или национальных, приводит к строгому
		наказанию. Процессы по поводу рекламаций могут быть очень разорительными.
		Предупреждение лучше, чем исправление, и фирмы обязаны выполнять
		законодательные требования и выдерживать высокие стандарты. Молоко и молочные
		продукты по своей природе являются идеальными средами для роста микроорганизмов,
		включая многие патогенные виды. Поэтому свод законов, относящихся к молоку и молочным
		продуктам — производству, упаковке, хранению, транспортировке, продаже, является
		наиболее объемным из применяемых для пищевых продуктов. В каждой стране имеются
		собственные национальные и иногда местные законодательные нормы.
	\end{flushleft}
	
	\begin{flushleft}
		\bfseries Объекты мойки
	\end{flushleft}
	
	\begin{flushleft}
		В отношении результатов мойки применяются следующие термины для определения
		степени чистоты:
	\end{flushleft}
	
	\begin{flushleft}
		\begin{itemize}
			\item Физическая чистота — удаление всех видимых следов загрязнений с поверхности
			\item Химическая чистота — удаление не только всех видимых загрязнений, но и микроскопических осадков, которые можно обнаружить по вкусу или запаху, но которые невидимы невооруженным глазом
			\item Бактериологическая чистота достигается дезинфекцией
			\item Стерильная чистота — уничтожение всех микроорганизмов.
		\end{itemize}
	\end{flushleft}
	
	\begin{flushleft}
		Важно отметить, что оборудование может быть бактериологически чистым и при этом не
		обязательно физически или химически чистым. Однако степень бактериологической чистоты
		легче достигнуть, если поверхности, по меньшей, мере будут являться физически чистыми.
		Практически всегда цель операции мойки на молочных предприятиях — достижение как
		химической, так и бактериологической степени чистоты. Следовательно, поверхности
		оборудования сначала тщательно очищаются химическими моющими средствами, а затем
		дезинфицируются.
	\end{flushleft}
	
	\begin{flushleft}
		\bfseries Загрязнения
	\end{flushleft}
	
	\begin{flushleft}
		Загрязнения представляют собой осадок, оставшийся на поверхности, и его состав в данном случае включает компоненты молока, которые потребляются бактериями, «спрятавшимися» в грязи.
	\end{flushleft}
	
	\begin{flushleft}
		\bfseries Нагреваемые поверхности
	\end{flushleft}
	
	\begin{flushleft}
		При нагревании молока свыше 60° С начинает образовываться молочный камень.
		Это осадок фосфатов кальция (и магния), белков, жира и т.д., который легко можно увидеть на пластинах теплообменника после длительного производственного цикла, в секции нагревания и в первой части секции регенерации. Осадок прочно прикипает к поверхностям, и после восьми или более часов работы его цвет меняется с беловатого до коричневатого.
	\end{flushleft}
	
	
	\begin{tabular}{|p{2.7cm}| p{3cm}| p{4cm} |p{4cm}|}
		\hline
		Компонент на & Растворимость & \multicolumn{2}{c|}{ Сложность удаления} \\
		
		поверхности&  & Низко/средне-температурная пастеризация & Высокотемпературная пастеризации/ВТО \\
		\hline
		Сахар & В воде & Легко & Карамелизация Трудно \\
		\hline
		Жир & Не в воде & Трудно В щелочи & Полимеризация Трудно \\
		\hline
		Белок & Не в воде & Очень трудно В щелочи Слегка в кислоте & Денатурация Очень трудно \\
		\hline
		Минеральные вещества & Различная в воде Большинство солей в кислоте & Различная & Различная \\
		\hline
	\end{tabular}
	
	\begin{flushleft}
		\bfseries Нагреваемые поверхности
	\end{flushleft}
	
	\begin{flushleft}
		При нагревании молока свыше 60° С начинает образовываться молочный камень.
		Это осадок фосфатов кальция (и магния), белков, жира и т.д., который легко можно увидеть на пластинах теплообменника после длительного производственного цикла, в секции нагревания и в первой части секции регенерации. Осадок прочно прикипает к поверхностям, и после восьми или более часов работы его цвет меняется с беловатого до коричневатого.
	\end{flushleft}
	
	\begin{flushleft}
		\bfseries Холодные поверхности
	\end{flushleft}
	
	\begin{flushleft}
		Пленка молока пристает к стенкам трубопроводов, насосов, резервуаров и т.п. («холодным» поверхностям). При опорожнении системы мойку следует начинать как можно скорее, иначе эта пленка высохнет, и удалить ее будет труднее.
	\end{flushleft}
	
	\begin{flushleft}
		\bfseries 	Процедура мойки
	\end{flushleft}
	
	\begin{flushleft}
			В свое время (а иногда и сейчас) мойка молочного оборудования производилась людьми, вооруженными щетками и моющими растворами, которым приходилось разбирать оборудование и влезать в танки, чтобы добраться до очищаемых поверхностей. Это было не только трудоемким, но и неэффективным мероприятием; продукты зачастую подвергались повторному загрязнению от неудовлетворительно вымытого оборудования.
	\end{flushleft}
	
	\begin{flushleft}
	Системы циркуляционной безразборной мойки (CIP) могут использоваться для различных частей технологической линии и позволяют производить высококачественную мойку, обеспечивая необходимое санитарное состояние.
	\end{flushleft}
	
	\begin{flushleft}
Операции мойки должны выполняться в строгом соответствии с тщательно разработанной процедурой с целью достижения требуемой степени чистоты.
	\end{flushleft}
	
	\begin{flushleft}
			Это означает, что в любое время последовательность операций должна быть одной и той же. Цикл мойки оборудования на молочном предприятии включает следующие стадии:
	\end{flushleft}
	
	\begin{itemize}
		\item Удаление остатков продуктов путем соскребания, слива и вытеснения водой или сжатым воздухом
		\item Предварительное ополаскивание водой для удаления рыхлых загрязнений
		\item Мойка с применением моющих средств
		\item Ополаскивание чистой водой
		\item Дезинфекция нагреванием или химическими средствами (по выбору); если эта стадия производится, то цикл завершается окончательной промывкой при соблюдении условия использования воды высокого качества.  
	\end{itemize}
	
	\begin{flushleft}
		Каждая стадия требует некоторого времени для достижения приемлемого результата.
		В таблице 1 приведены некоторые характеристики загрязнений и результаты воздействия на них химических средств.
		Как правило, мойка с применением щелочных моющих средств должна производиться при той же самой температуре, воздействию которой подвергался продукт, но не ниже 70° С.
	\end{flushleft}
	
	\begin{flushleft}
		\bfseries 	Удаление остатков продуктов
	\end{flushleft}
	
	\begin{flushleft}
			Все остатки продуктов в конце работы должны быть удалены из производственной линии.
		Это необходимо сделать по трем причинам:
	\end{flushleft}
	
	\begin{itemize}
		\item Для минимизации потерь продуктов
		\item Для облегчения мойки
		\item Для снижения нагрузки на систему канализации, что зачастую означает существенное снижение расходов на эксплуатацию канализационной системы
	\end{itemize}
	
	\begin{flushleft}
		Необходимо предоставить время для того, чтобы продукт стек со стенок резервуаров и трубопроводов Поверхности, покрытые затвердевшими остатками, например, в маслоизготовителях, необходимо отскребать начисто. Перед началом мойки оставшееся молоко вытесняется из производственной линии водой. Если это возможно, то молоко из трубопроводов выдувается или вымывается водой в сборные баки.
	\end{flushleft}
	
	\begin{flushleft}
		\bfseries Предварительное ополаскивание водой
	\end{flushleft}	
	
	\begin{flushleft}
		Предварительное ополаскивание водой должно проводиться немедленно после окончания производственного процесса. Иначе остатки молока высохнут и пристанут к поверхностям, что усложняет мойку Остатки молочного жира легче всего вымываются, если вода для ополаскивания теплая, но ее температура не должна превышать 55°С для предотвращения коагуляции белков.
	\end{flushleft}
	
	\begin{flushleft}
		Предварительное ополаскивание необходимо продолжать до тех пор, тюка вода, выходящая из системы, не будет чистой, так как остающиеся загрязнения повышают расход моющих средств и снижают активность хлора (если он присутствует) в моющем средстве. Если на поверхностях есть остатки засохшего молока, то замачивание оборудования будет целесообразным. Замачивание размягчает загрязнения и делает процесс мойки более эффективным.
	\end{flushleft}
	
	\begin{flushleft}
		Смесь молока и воды после предварительного ополаскивания собирают в бак для специальной переработки. Эффективная промывка удаляет не менее 90\% не засохших загрязнений, причем обычно 99\% от общего количества осадка.
	\end{flushleft}
	
	\begin{flushleft}
		\bfseries Мойка с применением моющих средств
	\end{flushleft}
	
	\begin{flushleft}
	Для обеспечения хорошего контакта между щелочным моющим раствором, обычно каустической содой (NaOH), и пленкой загрязнения необходимо добавлять смачивающий агент (смачиватель), который понижает поверхностное натяжение жидкости. Обычно используют Типол (алкиларилсульфонат) — один из анионных смачивателей.
	\end{flushleft}
	
	\begin{flushleft}
			Моющее средство должно обеспечивать диспергирование загрязнений и обволакивание частиц суспензии для предотвращения образования хлопьев. Эффективными эмульгирующими и диспергирующими агентами являются полифосфаты, которые также умягчают воду. Наиболее широко распространены трифосфат натрия и комплексные фосфатные соединения.
	\end{flushleft}
	 
	 	\begin{flushleft}
	 	Для обеспечения удовлетворительных результатов мойки данным моющим средством следует тщательно контролировать несколько параметров процесса. Этими параметрами являются:
	 \end{flushleft}
	 
	 \begin{itemize}
	 	\item Концентрация раствора моющего сродства
	 	\item Температура раствора моющего средства
	 	\item Механическое воздействие на очищаемую поверхность (скорость)
	 	\item Продолжительность мойки (время)
	 \end{itemize}
	 
	 \begin{flushleft}
	 	\bfseries 	Концентрация моющего раствора
	 \end{flushleft}	
	 
	 	\begin{flushleft}
	 		Количество моющего средства в растворе перед началом мойки должно быть доведено до требуемой концентрации. В процессе мойки раст вор разбавляется промывочной водой и остатками молока. Имеет место также и некоторая нейтрализация. Поэтому в процессе мойки необходимо проверять концентрацию. Если этого не делать, это может серьезно повлиять на результат. Проверка может осуществляться либо вручную, либо автоматически. Дозировка всегда должна производиться в соответствии с указаниями изготовителя моющего средства, так как увеличение концентрации не всегда улучшает эффективность мойки — в действительности можно получить обратный эффект из-за пенообразования и т.п. Использование слишком большого количества моющих средств делает очистку непозволительно дорогостоящей.
	 \end{flushleft}
	 
	 
	 \begin{flushleft}
	 	\bfseries 	Температура моющего раствора
	 \end{flushleft}	
	 
	 	\begin{flushleft}
	 	В общем случае эффективность воздействия раствора моющего средства возрастает с ростом температуры. У смеси моющих средств всегда имеется оптимальная температура, при которой ее следует использовать.
	 \end{flushleft}
	 
	 
	 	\begin{flushleft}
	 	Как правило, мойка раствором щелочи должна производиться при той же самой температуре, воздействию которой подвергался продукт, но не ниже 70°С. Для мойки раствором кислоты рекомендуется температура 68-70°С.
	 \end{flushleft}
	 
	 \begin{flushleft}
	 	\bfseries 	Эффективность механической мойки
	 \end{flushleft}	
	 
	 
	 	\begin{flushleft}
	 При ручной мойке для достижения требуемого механического воздействия используются щетки-скребки (рис.2). При механизированной мойке трубопроводов, резервуаров и другого технологического оборудования эффективность механическою воздействия определяется скоростью потока. Насосы, подающие моющее средство, должны иметь большую мощность, чем технологические насосы, обеспечивать скорость потока в трубах 1,5-3,0 м/с. При этих скоростях поток жидкости сильно турбулентен Это создает эффективное очищающее действие на поверхность оборудовании.
	 \end{flushleft}
	 
	 
	 
	 \begin{flushleft}
	 	\bfseries Продолжительность мойки
	 \end{flushleft}	
	 
	 	\begin{flushleft}
	 	Продолжительность этапа мойки с применением моющего средства должна быть тщательно рассчитана для достижения оптимальной эффективности. В то же самое время необходимо принимать во внимание стоимость электроэнергии, тепла, воды и рабочей силы. Недостаточно промыть трубопровод один раз моющим раствором.
	 \end{flushleft}
	 
	 
	 	\begin{flushleft}
	 	Для растворения всех загрязнений моющий раствор должен циркулировать в системе достаточно долго. Требующееся для этого время зависит от толщины отложений (и температуры моющего раствора). Пластины теплообменников с осадком, включающим коагулировавшие белки, необходимо промывать циркулирующим раствором азотной кислоты в течение 20 минут, тогда как для растворения слоя загрязнений на стенках танка для хранения молока при обработке щелочным раствором достаточно 10-минутной обработки.
	 \end{flushleft}
	 
	 \begin{flushleft}
	 	\bfseries 	Ополаскивание чистой водой
	 \end{flushleft}	
	 
	 
	 	\begin{flushleft}
	 	После циркуляции моющего раствора поверхности следует достаточно долго промывать водой для удаления всех следов моющего средства. Моющий раствор, остающийся в системе после мойки, может попасть в молоко. После ополаскивания во всех частях системы должен быть осуществлен дренаж.
	 \end{flushleft}
	
 \begin{flushleft}
		Для ополаскивания предпочтительно использовать умягченную воду.
\end{flushleft}	


\begin{flushleft}
	Это предупреждает появление накипи на очищаемых поверхностях. Жесткая вода с высоким содержанием солей кальция должна умягчаться с помощью ионообменных фильтров до значения жесткости 2-4°dH (немецкие градусы жесткости).
	Оборудование и трубопроводы после обработки концентрированными растворами щелочи и кислоты при высокой температуре являются практически стерильными.
\end{flushleft}

\begin{flushleft}
	Далее необходимо предотвратить развитие микроорганизмов в промывной воде в системе. Это осуществляется путем подкисления воды для окончательной промывки до рН менее 5 добавлением фосфорной или лимонной кислоты. Кислая среда предотвращает рост большинства бактерий.
\end{flushleft}

 \begin{flushleft}
	\bfseries 	Дезинфекция
\end{flushleft}	

 \begin{flushleft}
		Должным образом проведенная мойка раствором кислоты или щелочи позволяет достичь не только физическую и химическую степень чистоты, но и бактериологическую степень.
\end{flushleft}	

\begin{flushleft}
	Степень бактериологической чистоты может быть в дальнейшем повышена с помощью дезинфекции, после которой оборудование фактически не содержит бактерий. Для некоторых продуктов (УВТ-молоко, стерилизованное молоко) необходима стерилизация оборудования до полного удаления микроорганизмов с его поверхностей.
\end{flushleft}

 \begin{flushleft}
	Оборудование для производства молока может быть дезинфицировано следующими способами:
\end{flushleft}	

\begin{itemize}
	\item Термическая дезинфекция (кипящая вода, горячая вода, пар)
	\item Химическая дезинфекция (хлор, кислоты, йодсодержащие вещества, перекись водорода, и т.д.

\end{itemize}

\begin{flushleft}
	Дезинфекция может проводиться по утрам, непосредственно перед началом переработки молока. Молоко подается сразу же после слива дезинфицирующего раствора из системы.
\end{flushleft}
 
 \begin{flushleft}
 	Если дезинфекция осуществляется в конце рабочего дня, необходимо провести ополаскивание системы, чтобы остатки дезинфицирующего раствора не воздействовали на металлические поверхности.
 \end{flushleft}
 
 \begin{flushleft}
	\bfseries Системы безразборной мойки (CIP)
\end{flushleft}	

 \begin{flushleft}
		Безразборная мойка подразумевает циркуляцию промывочной воды и растворов моющих средств через емкости, трубопроводы и технологические линии без разборки оборудования. CIP можно определить как циркуляцию моющих жидкостей через машины и другое оборудование в контуре мойки. Поток жидкости, проходящий с высокой скоростью по поверхности оборудования, оказывает на него эффективное механическое воздействие, очищая от слоя загрязнений. Это применимо только к потокам в трубопроводах, теплообменниках, насосах, клапанах, сепараторах и т.д.
\end{flushleft}	

\begin{flushleft}
		Традиционным способом мойки резервуаров большой вместимости является распыление моющего средства на верхние части их поверхности с последующим произвольным отеканием вниз по стенкам. В этом случае часто не оказывается достаточное механическое воздействие, но эффективность может быть повышена при использовании специально предназначенных распылительных устройств, одно из которых представлено на рис.3. Для мойки танков требуются большие объемы моющего средства, которое должно циркулировать с высокой скоростью.
\end{flushleft}
 
 \begin{flushleft}
	\bfseries 	Контуры CIP
\end{flushleft}	

\begin{flushleft}
	Вопрос о типах оборудования, которое можно подвергать очистке в одном и том же контуре, определяется следующими факторами:
\end{flushleft}
 
 \begin{itemize}
 	\item Отложения остатков продукта должны быть одного и того же типа, так что можно использовать одни и те же моющие и дезинфицирующие средства
 	\item Очищаемые поверхности оборудования должны быть сделаны из одних и тех же материалов или, по крайней мере, из материалов, совместимых с одними и теми же моющими и дезинфицирующими средствами
 	\item Все элементы контура должны быть доступны для очистки в одно и то же время.
 \end{itemize}
 
 \begin{flushleft}
 	Таким образом, оборудование для переработки молока может быть прикреплено для проведения мойки к ряду контуров, мойка которых осуществляется в разное время.
 \end{flushleft}
 
\begin{flushleft}
	\bfseries 	Совместимые материалы и конструкция системы
\end{flushleft}	

\begin{flushleft}
		Для эффективной безразборной мойки проектируемое оборудование должно быть включено в контур мойки, а также быть легкодоступным для мойки. Все поверхности должны быть доступны для раствора моющего средства. Не должно быть тупиков, в которые не может проникнуть моющее средство или через которые оно не может циркулировать (см. рис.4). Машины и трубопроводы должны быть смонтированы таким образом, чтобы обеспечивать эффективный дренаж. Все карманы и ловушки, откуда остатки воды невозможно слить, представляют собой зоны для быстрого размножения бактерий и приводят к серьезному риску бактериального обсеменения продукта.
\end{flushleft}
 
\begin{flushleft}
Материалы технологического оборудования — например, нержавеющая сталь, пластмассы и резины — должны быть такого качества, чтобы предотвращать перенос запаха или привкуса в продукт. Они также должны быть способны противостоять воздействию моющего или дезинфицирующего средства при температурах мойки.
\end{flushleft}	

\begin{flushleft}
В некоторых случаях поверхности трубопроводов и оборудования могут подвергаться химическому воздействию и загрязнять продукт. Медь, латунь и олово чувствительны к воздействию концентрированных растворов кислот и щелочей. Даже незначительное количество (следы) меди в молоке приводит к окисленному вкусу (маслянистому или рыбному привкусу). Нержавеющая сталь является универсальным материалом для поверхностей, контактирующих с продуктом в современных установках для переработки молока. Следовательно, проблема возникновения металлических загрязнений обычно не возникает. Однако нержавеющая сталь может подвергаться воздействию растворов хлора.
\end{flushleft}
 
\begin{flushleft}
		Электролитическая коррозия обычно имеет место, если медные или латунные элементы встроены в системы из нержавеющей стали. В этих условиях риск загрязнения достаточно высок.
\end{flushleft}	

\begin{flushleft}
Электролитическая коррозия также может возникнуть в системе, изготовленной из разных типов стали, мойка которой осуществляется катионактивными средствами.
\end{flushleft}
 
\begin{flushleft}
	Эластомеры (например, резиновые прокладки) могут подвергаться воздействию хлора и окислителей, что вызывает их растрескивание или разрушение и приводит к попаданию частиц резины в молоко.
\end{flushleft}	

\begin{flushleft}
	Пластмассовые составляющие технологического оборудования могут являться источником загрязнения. Компоненты некоторых пластмасс могут растворяться в молочном жире. Моющие растворы также могут оказывать подобное действие. Пластмассы, используемые в оборудовании для молочной промышленности, должны удовлетворять определенным критериям в отношении их состава и устойчивости.
\end{flushleft}
 
\begin{flushleft}
	\bfseries 	Программы систем CIP
\end{flushleft}	

\begin{flushleft}
	Программы CIP мойки оборудования для переработки молока различаются в зависимости от наличия или отсутствия нагреваемых поверхностей в контурах, предназначенных для мойки. Мы различаем:
\end{flushleft}
 
 \begin{itemize}
 	\item Программы CIP для контуров, включающих пастеризаторы и другие виды оборудования, имеющего нагреваемые поверхности (ВТО и т.п.)
 	\item Программы CIP для контуров, включающих трубопроводы, резервуары и другое технологическое оборудование, не имеющее нагреваемых поверхностей.
 \end{itemize}
 
\begin{flushleft}
	Основное различие между этими двумя типами состоит в том. что циркуляция кислоты всегда должна быть включена в программы первого типа с целью удаления осажденных белков и солей с поверхностей оборудования для тепловой обработки. Программа CIPдля пастеризатора (“горячего элемента») может состоять из следующих стадий:
\end{flushleft}	

\begin{enumerate}
	\item Ополаскивание теплой водой в течение 10 минут
	\item Циркуляция раствора щелочною моющего средства (0,5-1.5\%) продолжительностью около 30 минут при температуре 75°С
	\item Ополаскивание теплой водой в течение 5 минут для удаления остатков раствора щелочи
	\item Циркуляция раствора (азотной) кислоты (0,5-1\%) продолжительностью около 20 минут при температуре 70°С
	\item Ополаскивание холодной водой
	\item  Постепенное охлаждение холодной водой продолжительностью около 8 минут. Дезинфекция пастеризаторов обычно проводится утром, перед началом производства Как правило, это осуществляется циркуляцией горячей воды при 90-95° С в течение 10-15 минут после достижения температуры потока на возврате, по меньшей мере, 85 С.
\end{enumerate}

\begin{flushleft}
		На некоторых установках после промывки водой система CIP сначала запускает программу мойки раствором кислоты для удаления отложений солей и разрушения таким образом слоя загрязнений с целью облегчения растворения белков при последующей щелочной очистке. Если дезинфекция осуществляется с применением хлорсодержащих средств, существует опасность возникновения коррозии в случае неполною удаления раствора кислотного моющего средства. Следовательно, если начинать с мойки раствором щелочи и заканчивать раствором кислоты после промежуточной промывки водой, то установку необходимо промыть слабым раствором щелочи для нейтрализации кислоты перед началом дезинфекции с помощью хлорсодержащих средств.
\end{flushleft}
 
\begin{flushleft}
	Стадии для CIP холодных элементов:
\end{flushleft}	

\begin{enumerate}
	\item Ополаскивание водой
	\item Циркуляция щелочного моющего средства
	\item Ополаскивание водой
	\item Дезинфекция горячей водой
	\item Охлаждение холодной водой
\end{enumerate}

\begin{flushleft}
	Программа CIP для контуров трубопроводов, танков и других “холодных элементов” может включать следующие стадии:
\end{flushleft}

\begin{enumerate}
	\item Ополаскивание теплой водой в течение 3 минут
	\item Циркуляция 0,5-1,5 \%-ного раствора щелочного моющего средства при температуре 75° С не больше 10 минут
	\item Ополаскивание теплой водой в течение 3 минут
	\item Дезинфекция горячей водой 90-95°С в течение 5 минут
	\item Постепенное охлаждение холодной водой примерно в течение 10 минут (для танков охлаждение обычно не требуется).
\end{enumerate}

\begin{flushleft}
	\bfseries 	Проектирование систем CIP
\end{flushleft}	

\begin{flushleft}
	На практике не существует ограничений для удовлетворения индивидуальных требований к размерам и степени сложности установок для CIP.
\end{flushleft}
 
\begin{flushleft}
Установка для CIP на молочном заводе включает оборудование, необходимое для хранения, транспортировки и распределения моющих растворов в различные контуры CIP. Точная схема системы определяется множеством факторов, например:
\end{flushleft}	

\begin{itemize}
	\item Количеством индивидуальных контуров CIP, обслуживаемых системой. Сколько из них «горячих» и сколько «холодных»?
	\item Собираются ли остатки вытесненного молока? Будет ли оно перерабатываться (выпариваться)?
	\item Какой метод дезинфекции используется? Посредством химических средств, пара или горячей воды?
	\item Предусмотрено ли однократное использование моющих растворов или они повторно используются?
	\item  Каково примерное количество потребляемого пара, мгновенное и общее, для мойки и стерилизации?
\end{itemize}

\begin{flushleft}
	Оглядываясь назад, на историю CIP, мы обнаруживаем две концептуальные школы:
\end{flushleft}
 
 \begin{enumerate}
 	\item Централизованная мойка
 	\item Децентрализованная мойка.
 \end{enumerate}
 
\begin{flushleft}
	До конца пятидесятых годов мойка была децентрализованной. Оборудование доя мойки размещалось на заводе, вблизи от технологическою оборудования. Наведение моющих растворов требуемой концентрации проводилось вручную — неприятная и опасная процедура для обслуживающего персонала. Потребление моющих средств было высоким, что делало мойку дорогостоящей процедурой.
	Централизованные системы CIP были разработаны в шестидесятые и семидесятые годы.
\end{flushleft}	

\begin{flushleft}
	На заводах были установлены централизованные системы мойки CIPВода для промывки, нагретые растворы моющих средств и горячая вода подавались из этой установки по системе трубопроводов во все контуры CIP завода. Затем использованные растворы поступали обратно на центральную станцию в соответствующие сборные емкости. После чего проводилось восстановление требуемой концентрации растворов. Моющие растворы использовались повторно до тех пор, пока не загрязнялись, в противном случае дальнейшему использованию не подлежали.
\end{flushleft}

\begin{flushleft}
		Централизованные CIP хорошо работают на многих молочных комбинатах, но на крупных заводах коммуникации между центральной станцией CIP и периферийными контурами CIP становятся слишком протяженными. Трубопроводы системы CIPсодержат большие объемы жидкостей даже после дренажа системы. Вода, остающаяся в трубах после промывки, разбавляет моющие растворы, что означает необходимость дополнительного введения значительного количества моющих средств для поддержания необходимой концентрации. Чем больше расстояние, тем больше расходы на мойку. Поэтому на крупных заводах в конце семидесятых годов наблюдается обратная тенденция к станциям децентрализованной мойки. В каждом подразделении имеется собственная система CIP. 
\end{flushleft}	

\begin{flushleft}
	\bfseries 	Централизованная CIP
\end{flushleft}	

\begin{flushleft}
Централизованные системы используются главным образом на небольших молочных заводах с относительно короткими линиями коммуникаций.
\end{flushleft}

\begin{flushleft}
	Вода и моющие растворы подаются насосами из танков для хранения центральной станции в различные контуры CIP.
	Моющие растворы и горячая вода хранятся в нагретом виде в изолированных танках. Требуемая температура поддерживается теплообменниками. Вода для окончательного ополаскивания собирается в танк для промывочной воды и используется в качестве воды для предварительного
\end{flushleft}

\begin{flushleft}
После многократного использования моющие растворы содержат значительное количество загрязнений и сливаются. Затем проводится мойка танка для хранения и заполнение свежеприготовленными растворами. Важно для предотвращения опасности загрязнения чистой технологической линии через регулярные промежутки времени опорожнять и промывать танки для хранения воды, особенно танк для промывочной воды.
\end{flushleft}	

\begin{flushleft}
	Станция подобного типа обычно имеет высокую степень автоматизации. В танках имеются электроды для отслеживания верхнего и нижнего предельных уровней. Возврат моющих растворов контролируется датчиками проводимости. Проводимость пропорциональна концентрации растворов, которая обычно используется для мойки оборудования для производства молока. На стадии промывки водой концентрация моющего раствора постепенно понижается. При достижении заданного уровня клапан переключается и направляет жидкость в дренаж вместо соответствующего танка для моющего средства. Программы CIPуправляются компьютерным контроллером последовательности операций. Большие станции CIP для обеспечения необходимой производительности могут быть оборудованы многосекционными танками.
\end{flushleft}
	      
   \begin{flushleft}
   	\bfseries 	Децентрализованная CIP
   \end{flushleft}
   
   \begin{flushleft}
   		Децентрализованная CIP является привлекательной альтернативой для крупных молочных заводов, где расстояние между центральной станцией CIP и периферийными контурами CIP исключительно велико. Большая станция CIP заменяется несколькими малыми узлами, расположенными поблизости от различных групп технологического оборудования молочного завода.
%   	На рис.7 показана схема децентрализованной системы CIP. Она также имеет центральную станцию для хранения растворов щелочи и кислоты, которые индивидуально распределяются по отдельным узлам CIP в основных линиях. Подача и нагрев воды для ополаскивания (и раствора кислоты в случае необходимости) обеспечиваются локально периферийными станциями, одна из которых показана на рис.8.
   \end{flushleft}
   
   \begin{flushleft}
   		Эти станции работают в соответствии с принципом использования минимально необходимого тщательно отмеренного объема жидкости для каждой стадии программы мойки — ровно столько, сколько нужно для заполнения данного контура мойки.
   \end{flushleft}
   
   \begin{flushleft}
   	Для нагнетания моющего раствора в контур с высокой скоростью используется мощный циркуляционный насос.
   \end{flushleft}
	
	\begin{flushleft}
		Мойка, основанная на циркуляции небольших объемов моющих растворов, имеет свои преимущества. Как мгновенное, так и общее потребление воды и пара можно значительно сократить. В этом случае вытесненные после первого ополаскивания остатки молока более концентрированы, следовательно, их легче транспортировать и дешевле выпаривать. 
	\end{flushleft}
	
	\begin{flushleft}
		Децентрализованные CIP снижают нагрузку на канализационные системы в сравнении с централизованными CIP, использующими большие объемы жидкостей.
	\end{flushleft}
	
	\begin{flushleft}
		Концепция однократного использования моющих растворов была внедрена в связи с появлением децентрализованных CIP в противовес обычной практике многократного использования растворов в централизованных системах. Идея однократного использования основана на предположении, что состав моющего раствора можно оптимизировать для данного контура. После однократного использования раствор считается отработанным. Однако в некоторых случаях он может быть использован в последующей программе для предварительного ополаскивания.
	\end{flushleft}
	
	\begin{flushleft}
		\bfseries 	Контроль эффективности мойки
	\end{flushleft}
	
	\begin{flushleft}
			Проверка эффективности мойки должна рассматриваться как важная часть операций по мойке. Существуют два типа этой проверки: визуальная и бактериологическая. Вследствие развития автоматизации современные технологические линии редко доступны для визуального осмотра.
	\end{flushleft}
	
	\begin{flushleft}
		Поэтому визуальный контроль необходимо заменять бактериологическим, сконцентрированным в нескольких критических точках линии. Результаты CIP обычно проверяются путем культивирования бактерий группы кишечных палочек. Качество мойки считается удовлетворительным, если в результате контроля исследуемой поверхности площадью 100 см2 присутствует менее 1 бактерии труппы кишечных палочек.
	\end{flushleft}
	
\begin{flushleft}
	Если количество бактерий превышает это значение, качество мойки считается неудовлетворительным. Этот контроль осуществляется на поверхности оборудования после завершения программы CIP. Это применимо к танкам и трубопроводам, особенно при обнаружении исключительно большого количества бактерий в продуктах. Образцы часто отбираются из воды после окончательного ополаскивания или первого продукта, прошедшего через линию после ее мойки.
\end{flushleft}

\begin{flushleft}
		Для обеспечения полного контроля качества процесса производства проводится микробиологический контроль упакованных продуктов. Программа полного контроля качества — в дополнение к анализу содержания бактерий группы кишечных палочек — включает также определение общего числа микроорганизмов и контроль органолептических показателей (дегустацию).
\end{flushleft}


	
	\end{document}