\sectionbreak \section*{ 
    \gostTitleFont
    \redline
    ВВЕДЕНИЕ
}
\titlespace

{\gostFont

\par \redline 	В последнее десятилетие нейронные сети пользуются большим спросом и привлекают немалое внимание публики. И хотя складывается впечатление, что нейронные сети {--} это новая и крайне прогрессивная технология, в действительности это одно из старейших направлений в информатике/кибернетике, сопоставимое по возрасту с операционными системами и старше, чем такие обыденные вещи, как работа с графикой и даже обычный графический интерфейс пользователя.

\par \redline 	И несмотря на то, что нейронные сети появились довольно давно, в действительности их потенциал стал раскрываться лишь в последнее десятилетие, когда стали активно развиваться такие вещи, как видеокарты и кратно возросла мощность компьютеров. В последние 2 года нейронные сети совершили как качественный, так и количественный скачок. Результатом чего явилась возможность использовать нейронные сети в таких, традиционно человеческих сферах, как рисование, генерирование музыки, ответы на вопросы человека с поразительной точностью и многое другое. 


\par \redline И хотя большая часть восторгов широкой публики направлена на вышеописанные способы применения, существует не мало сфер применения, которые ускользают от взора широкой публики. Одной из таких сфер применения является промышленность. Применение нейронных сетей в промышленности является потенциально полезной задумкой. Тем не менее, современные нейронные сети пишутся на языках высокого уровня и непригодны для использования в промышленности. Для полноценного внедрения нейронных сетей в промышленность, необходимо писать их на языках более низкого уровня, чем самый популярный язык для нейронных сетей python.


\par \redline Данная работа посвящена задаче поиска аномалий в данных технологического процесса производственного оборудования, а именно моечной станции. Для выполнения данной задачи будут использоваться нейронные сети, поскольку решение задачи поиска аномалий нейросетевым способом, на сегодняшний день, является одним из самых актуальных, адаптивных и перспективных способов решения данной задачи [14{--}18]. Данная работа является совокупностью задач прогнозирования, поиска аномалий и машинного обучения, поскольку в её рамках будет рассматриваться изучение предметной области и определение задач и целей, сбор, анализ и обработка данных технологического процесса моечной станции, их подготовка к необходимым задачам и обучению модели нейронной сети, проектирование и реализация нейронной сети, её обучение, тестирование и оценка, в конечном итоге готовую модель нейронной сети необходимо развернуть для взаимодействия с оборудованием.
\par
}
